\documentclass[twoside]{book}
\newif\ifpdf
\ifx\pdfoutput\undefined
\pdffalse % we are not running PDFLaTeX
\else
\pdfoutput=1 % we are running PDFLaTeX
\pdftrue
\fi
%\usepackage{times}
%\usepackage{amsmath}
\usepackage{calc}
\usepackage[latin1]{inputenc}
\usepackage{FFF}
\usepackage{amsfonts}
\usepackage{amsmath}
\usepackage{hyperref}
\usepackage{FFF}
\usepackage{makeidx}
\usepackage{color}
\usepackage{multicol}
\usepackage{graphicx}
%\usepackage{dessin}
\topmargin -1.54cm
\oddsidemargin 0cm   %marge a 2 cm
\evensidemargin 0cm  %marge a 2 cm
\newcommand{\indente}{\hbox to \parindent {\hss}}
\parindent 0cm
\headsep 0.5cm
\topskip .5cm
\footskip 1cm
\headheight 1.0cm
\textwidth  16.5cm
%  \parindent 0cm
\textheight 24cm
\def\freefempp{\texttt{freefem++ }}
\def\textRed{\color{red}}
\def\textBlack{\color{black}}
\def\Blue#1{\textcolor{blue}{#1}}
\def\Black#1{\textcolor{black}{#1}}
\def\Red#1{\textcolor{red}{#1}}
\def\Magenta#1{\textcolor{magenta}{#1}}
\def\hin{\hbox{ in }}
\def\hon{\hbox{ on }}
\def\Cpp{\texttt{C++~}}
\def\R{\mathrm{I\!R}}
\def\example{\textbf{Example:}}
\def\eq#1{\Blue{\[#1\]}}
\def\R{\mathbb{R}}
\def\Z{\mathbb{Z}}
\def\itemtt[#1]{ \item[\texttt{#1}]}
\def\plot[#1]#2#3{\begin{figure}[hbt]
\begin{center}
    \includegraphics*[#1]{#2}
\end{center}
\caption{\label{#2} #3}
\end{figure}
}
\def\Ostream{\texttt{ostream}}
\def\Istream{\texttt{istream}}
\def\Bool{\texttt{bool}}
\def\Real{\texttt{real}}
\def\Int{\texttt{int}}
\def\vecttwo#1#2{\left|\begin{smallmatrix} #1 \\ #2 \end{smallmatrix}\right.}
\def\vdeux(#1,#2){\left|\begin{smallmatrix} #1 \\ #2 \end{smallmatrix}\right.}
\def\HLINE#1{\hbox to \hsize {#1}}
\def\twoplot[#1]#2#3#4#5{
\begin{figure}[hbt]
\begin{multicols}{2}
\begin{center}
    \includegraphics*[#1]{#2}
    \caption{\label{#2} #4}
\end{center}
\begin{center}
    \includegraphics*[#1]{#3}
    \caption{\label{#3} #5}
\end{center}
\end{multicols}
\end{figure}
}% end twoplot macro
\newtheorem{remark}{\textbf{Remark}}
\newtheorem{bug}{\textbf{Bug:}}
\newtheorem{proposition}{\textbf{Proposition}}
\newtheorem{algorithm}{\textbf{Algorithm}}
\newenvironment{ttlist}
   {\begin{list}{}{\renewcommand{\makelabel}[1]{\texttt{##1}\hfil}%
        \setlength{\labelwidth}{3cm}
        \setlength{\leftmargin}{\labelwidth+\labelsep}
    }}%
   {\end{list}}


\begin{document}
\graphicspath{{./}{plots/}}
\ifpdf
\DeclareGraphicsExtensions{.pdf, .jpg, .tif}
\else
\DeclareGraphicsExtensions{.eps,.ps, .jpg}
\fi

\let\subsubsection\subsection
\let\subsection\section
\let\section\chapter

%\subsection{Periodic }
\index{periodic}\index{fespace!periodic=}
Solve of the Laplace equation 
$$ -\Delta u= sin(x+\pi/4.)*cos(y+\pi/4.)$$ on 
a  square $]0,2\pi[^2$ with bi-periodic boundary condition.

\bFF

mesh Th=square(10,10,[2*x*pi,2*y*pi]);
// defined the \bgroup\tt fespace\egroup  with periodic condition
//    label :  2 and 4  are left and right   side with y the curve abcissa 
//             1 and 2  are bottom and upper side with x the curve abcissa
fespace Vh(Th,P2,periodic=[[2,y],[4,y],[1,x],[3,x]]);   
 Vh uh,vh;              // unkown and test function. 
 func f=sin(x+pi/4.)*cos(y+pi/4.);      //  right hand side function 

 problem laplace(uh,vh) =                      //  definion of  the problem 
    int2d(Th)( dx(uh)*dx(vh) + dy(uh)*dy(vh) ) //  bilinear form
  + int2d(Th)( -f*vh )                         //  linear form
;                

  laplace; // solve the problem plot(uh); // to see the result
  plot(uh,ps="period.eps",value=true);
\eFF

\plot[height=6cm]{period}{The isovalue of solution $u$ with periodic boundary condition}

An over exemple is in \texttt{periodic4.edp} file, with 
periodic condition no parallel to the axis.

Solve a diamond with a hole:
\bFF

real r=0.25;
// a diamond with a hole
border a(t=0,1){x=-t+1; y=t;label=1;}; 
border b(t=0,1){ x=-t; y=1-t;label=2;};
border c(t=0,1){ x=t-1; y=-t;label=3;};
border d(t=0,1){ x=t; y=-1+t;label=4;};
border e(t=0,2*pi){ x=r*cos(t); y=-r*sin(t);label=0;};
int n = 10;
mesh Th= buildmesh(a(n)+b(n)+c(n)+d(n)+e(n)); 
plot(Th,wait=1);
real r2=1.732;
func abs=sqrt(x^2+y^2);
//  warning for periodic condition: \hfilll
//  side a and c \hfilll
//  on side a (label 1) $ x \in [0,1] $ or $ x-y\in [-1,1] $ \hfilll
//  on side c (label 3) $ x \in [-1,0]$ or $ x-y\in[-1,1] $\hfilll
// so the common abcissa can be repectively $x$ and $x+1$
// or you can can try curviline abcissa $x-y$ and $x-y$ 
//  1 first way \hfilll
// fespace Vh(Th,P2,periodic=[[2,1+x],[4,x],[1,x],[3,1+x]]);  \hfilll    
// 2 second way \hfilll
 fespace Vh(Th,P2,periodic=[[2,x+y],[4,x+y],[1,x-y],[3,x-y]]);     

 Vh uh,vh;             

 func f=(y+x+1)*(y+x-1)*(y-x+1)*(y-x-1);                
 real intf = int2d(Th)(f);
 real mTh = int2d(Th)(1);
 real k =  intf/mTh; 
 cout << k << endl; 
 problem laplace(uh,vh) =                     
    int2d(Th)( dx(uh)*dx(vh) + dy(uh)*dy(vh) ) + int2d(Th)( (k-f)*vh ) ;                
 laplace; 
 plot(uh,wait=1,ps="perio4.eps"); 
\eFF

\plot[height=6cm]{perio4}{The isovalue of solution $u$ for
$ \Delta u = ((y+x)^{2}+1)((y-x)^{2}+1) - k$, in $\Omega$ and $\partial_{n} u =0 $ on hole,and with two periodic boundary condition on external border}



\end{document}