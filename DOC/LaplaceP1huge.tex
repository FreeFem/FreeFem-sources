\documentclass[twoside]{book}
\newif\ifpdf
\ifx\pdfoutput\undefined
\pdffalse % we are not running PDFLaTeX
\else
\pdfoutput=1 % we are running PDFLaTeX
\pdftrue
\fi
%\usepackage{times}
%\usepackage{amsmath}
\usepackage{calc}
\usepackage[latin1]{inputenc}
\usepackage{FFF}
\usepackage{amsfonts}
\usepackage{amsmath}
\usepackage{hyperref}
\usepackage{FFF}
\usepackage{makeidx}
\usepackage{color}
\usepackage{multicol}
\usepackage{graphicx}
%\usepackage{dessin}
\topmargin -1.54cm
\oddsidemargin 0cm   %marge a 2 cm
\evensidemargin 0cm  %marge a 2 cm
\newcommand{\indente}{\hbox to \parindent {\hss}}
\parindent 0cm
\headsep 0.5cm
\topskip .5cm
\footskip 1cm
\headheight 1.0cm
\textwidth  16.5cm
%  \parindent 0cm
\textheight 24cm
\def\freefempp{\texttt{freefem++ }}
\def\textRed{\color{red}}
\def\textBlack{\color{black}}
\def\Blue#1{\textcolor{blue}{#1}}
\def\Black#1{\textcolor{black}{#1}}
\def\Red#1{\textcolor{red}{#1}}
\def\Magenta#1{\textcolor{magenta}{#1}}
\def\hin{\hbox{ in }}
\def\hon{\hbox{ on }}
\def\Cpp{\texttt{C++~}}
\def\R{\mathrm{I\!R}}
\def\example{\textbf{Example:}}
\def\eq#1{\Blue{\[#1\]}}
\def\R{\mathbb{R}}
\def\Z{\mathbb{Z}}
\def\itemtt[#1]{ \item[\texttt{#1}]}
\def\plot[#1]#2#3{\begin{figure}[hbt]
\begin{center}
    \includegraphics*[#1]{#2}
\end{center}
\caption{\label{#2} #3}
\end{figure}
}
\def\Ostream{\texttt{ostream}}
\def\Istream{\texttt{istream}}
\def\Bool{\texttt{bool}}
\def\Real{\texttt{real}}
\def\Int{\texttt{int}}
\def\vecttwo#1#2{\left|\begin{smallmatrix} #1 \\ #2 \end{smallmatrix}\right.}
\def\vdeux(#1,#2){\left|\begin{smallmatrix} #1 \\ #2 \end{smallmatrix}\right.}
\def\HLINE#1{\hbox to \hsize {#1}}
\def\twoplot[#1]#2#3#4#5{
\begin{figure}[hbt]
\begin{multicols}{2}
\begin{center}
    \includegraphics*[#1]{#2}
    \caption{\label{#2} #4}
\end{center}
\begin{center}
    \includegraphics*[#1]{#3}
    \caption{\label{#3} #5}
\end{center}
\end{multicols}
\end{figure}
}% end twoplot macro
\newtheorem{remark}{\textbf{Remark}}
\newtheorem{bug}{\textbf{Bug:}}
\newtheorem{proposition}{\textbf{Proposition}}
\newtheorem{algorithm}{\textbf{Algorithm}}
\newenvironment{ttlist}
   {\begin{list}{}{\renewcommand{\makelabel}[1]{\texttt{##1}\hfil}%
        \setlength{\labelwidth}{3cm}
        \setlength{\leftmargin}{\labelwidth+\labelsep}
    }}%
   {\end{list}}

\title{
\DeclareFixedFont{\TitreFont}{\encodingdefault}{pnc}{r}{\shapedefault}{80pt}
 {\Blue{ \TitreFont Freefem++ \\ \vglue 1cm  Manual}} \\ \vglue 5cm  ~ \\  
      \normalsize  { version 1.34  \Red{(Under construction)} }
 \\ \vglue 1cm
 \Large \url{http://www.freefem.org} \\
\url{http://www.ann.jussieu.fr/\string~hecht/freefem++.htm} 
}
\author{author}

\makeindex
\begin{document}
\graphicspath{{./}{plots/}}
\ifpdf
\DeclareGraphicsExtensions{.pdf, .jpg, .tif}
\else
\DeclareGraphicsExtensions{.eps,.ps, .jpg}
\fi
\maketitle
\tableofcontents
\let\subsubsection\subsection
\let\subsection\section
\let\section\chapter

\huge

Consider the problem
\begin{equation}
 -\Delta u = f \hbox{ in } \Omega=\{(x,y)\in R^2 : 0 \le x \le 1 , 0 \le y \le 1\}
\end{equation}
\begin{equation}
u=0 \hbox{ on } \Gamma_2 \cup \Gamma_3 \cup \Gamma_4
\end{equation}
\begin{equation}
\frac{\partial u}{\partial n} +u = 1 \hbox{ on } \Gamma_1 
\end{equation}

The problem is solved by the finite element method, namely:
\\
Find $u\in V$ the space of continuous \index{femp1}piecewise linear functions
on a triangulation of $\Omega$ 
\begin{equation}
    \int_\Omega \nabla u\cdot\nabla v + \int_{\Gamma_1} uv - \int_{\Gamma_1} v -  \int_\Omega fv  = 0
\end{equation}
The first thing to do is to prepare the mesh (i.e. the triangulation) ;
that is done by first defining the border\index{border} (the sqare) with labels one two three and four \index{label} and then call the mesh generator \index{buildmesh}(buildmesh) with
the right orientation of the \index{border}border (by definition $\Omega$ is
on the left side of the oriented $\Gamma$, the ).

\bFF
\huge
mesh Th = square(10,10);
\eFF
\huge
The second thing is to define the continuous piecewise linear functions spaces.

\index{fespace}
\bFF
\huge
fespace Vh(Th,P1);  // define the Vh space
Vh uh,vh;  //  introduice the function and test function
func f=1;
func g=0;
\eFF
\huge
Next, \texttt{freefem++} will define
the PDE discretized by FEM in variational form with the following
instruction,  and solve the problem

\bFF
\huge
  problem laplace(uh,vh,solver=GMRES,tgv=1e5) =  // definition of the problem
    int2d(Th)( dx(uh)*dx(vh) + dy(uh)*dy(vh) )     //  bilinear form
  + int1d(Th,1)( uh*vh )                          //  linear form
  - int1d(Th,1)( vh )
  -int2d(Th)( f*vh )
  + on(2,3,4, uh=g) ;                                // boundary condition

laplace;  // solve the problem
plot(uh,ps=''Laplace.eps'',value = true)
\eFF
\huge
Next we can check that the result is correct.
Here we display the result 
\plot[height=10cm]{Laplace}{The isovalue of solution $u$}

\end{document}