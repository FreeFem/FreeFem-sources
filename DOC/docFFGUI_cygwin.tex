\documentclass[a4paper]{report}
\usepackage{graphicx}

\begin{document}

\section{Installation of Cygwin}
First you have to go to the site:\\
http://www.cygwin.com \\
Click on ``Install or Update now''\\
A window appears. \\
Click on ``Open'' then ``Next'' and then choose ``Install from Internet''\\
Click on ``Next'' twice and choose ``Direct Connection''.\\
Then choose one of the download sites.\\
For instance : ftp://ftp-stud.fht-esslingen.de\\
Then choose in each category the options to install. \\
(click on the symbol + for each, you can then see the options appear)\\
\\
Here are the required options for FreeFem++ and TCL TK to work :\\
+ all options of ``Devel'' (it includes gcc ...etc...)\\
+ all options of ``Graphics'' (specially ``Gnuplot'' to be able \\
to see the results of FreeFem++ on a graphic)\\
+ all options of X11 and XFree86 \\
\\
You can install ``Xemacs'' and others editors in the category ``Editors''.\\
\\
Then click on ``Next'' and wait that Cygwin is installed.\\
Then an ic�ne appears on yours Desktop.\\
Click on it and the Cygwin window is opened. It is like an Unix terminal.\\
\\
To work on a terminal X type : \\
-$>$ startX\\
You will have to work on a terminal X to have Gnuplot and visualize \\
the results of FreeFem++ scripts)\\
\\
Now you have to compile and install FreeFem++.\\
\\
\section{Compilation and Installation of FreeFem++ under Cygwin}
go to the site :\\
http://www.freefem.org\\
click on FreeFem++\\
and download FreeFem++ (the version when I wrote this is 1.38)\\
Choose to download in your cygwin/home/you directory.\\
\\
Then on your terminal Cygwin type :\\
-$>$ tar zxvf freefem++.tgz\\
to uncompress this file.\\
\\
Go into FreeFem++v1.38 directory by typing:\\
-$>$ cd FreeFem++v1.38\\
\\
Now you have to compile FreeFem++.\\
type:\\
-$>$ make all HOSTTYPE=i-386\\
to compile FreeFem++\\
If there is an error : ``... -ldl : no such file or directory''\\
Then you have to modify the Makefile-i386 which is in the directory src:\\
-$>$ cd src \\
Edit it (with xemacs for example):\\
-$>$ xemacs Makefile-i386\\
at line 1 : replace ``LIBLOCAL = -ldl'' by ``\#LIBLOCAL = -ldl''\\
It will comment this line because -ldl is not on your machine.\\
Then return to the main directory\\
-$>$ cd\\
and type:\\
-$>$ make all HOSTTYPE=i-386\\
\\
At the end of compilation, a directory called ``c-i386'' is created.\\
In this directory you can find the binary FreeFem++.\\
\\
You can now run an example:\\
First open an X terminal:\\
-$>$ startX\\
In this terminal go in to FreeFem++v1.38:\\
-$>$ cd FreeFem++v1.38\\
and type:\\
-$>$c-i386/FreeFem++ examples++-tutorial/adapt.edp\\
You can then see the gnuplot window with the graphical results.\\
\\
Now if you want to use the Graphical User Interface of FreeFem++\\
(called FFGUI)\\
you have to install the language TCL TK in which FFGUI has been written.\\
The version which works under cygwin is tcl8.4.0 and tk8.4.0\\
The latest version when I wrote this is tcl8.4.6 and tk8.4.6\\
\\
DON'T USE IT \\
\\
It works under Linux and MacOsX but not under Cygwin.\\
\\
You have to download tcl8.4.0 and tk8.4.0 :\\
For instance, go to Google.fr and type download tcl tk 8.4.0\\
And choose the ``Sourceforge.net: Project Filelist''.\\
Choose :\\
tcl8.4.0-src.tar.gz and tk8.4.0-src.tar.gz\\
When the download is finished you have to uncompress these directories:\\
-$>$ tar zxvf tcl8.4.0-src.tar.gz\\
-$>$ tar zxvf tk8.4.0-src.tar.gz\\
\\
\\
tcl8.4.0 and tk8.4.0 will work under Cygwin only if you apply \\
a patch on both:\\
These patches are on the site:\\
http://www.xraylith.wisc.edu/~khan/software/tcl\\
Choose ``Tcl/Tk8.4.0 for Cygwin\\
Click on ``very preliminary Cygwin ports of Tcl/Tk8.4.0\\
You are then on the site ftp\\
Follow the instructions of the README or follow these instructions:\\
\\
1) Run :\\
-$>$ xemacs tcl-8.4.0-cygwin.diff\\
By doing this, you create a new file called ``tcl-8.4.0-cygwin.diff''\\
on the site ftp click on ``tcl-8.4.0-cygwin.diff''\\
Do a Copy/Paste of the contain into your xemacs window and save it.\\
\\
2) Do the same with ``tk-8.4.0-cygwin.diff''\\
\\
Now you have to apply the patch in tcl8.4.0 and tk8.4.0\\
The two previous patch files (.diff) must be respectively \\
in tcl8.4.0 and tk8.4.0 directories.\\
-$>$ cp tcl-8.4.0-cygwin.diff tcl8.4.0\\
-$>$ cp tcl-8.4.0-cygwin.diff tk8.4.0\\
(If the two files are one level under tcl8.4.0 and tk8.4.0)\\
\\
Now apply the patches:\\
type:\\
\\
-$>$ cd tcl8.4.0\\
-$>$ patch -p0 -s $<$ tcl-8.4.0-cygwin.diff\\
\\
-$>$ cd
\\
-$>$ cd tk8.4.0\\
-$>$ patch -p0 -s $<$ tk-8.4.0-cygwin.diff\\
\\
Now you can compile and install TCL TK under Cygwin:\\
\\

\section{Compilation and Installation of tcl8.4.0 and  tk8.4.0 under Cygwin}
\section{compilation and installation of tcl8.4.0}
Go in to the directory tcl8.4.0/win\\
-$>$ cd tcl8.4.0\\
-$>$ cd win\\
Then type :\\
-$>$ ./configure\\
The two steps remaining are make and make install\\
type\\
-$>$ make\\
The compilation starts, when finished install by typing:\\
-$>$ make install\\
\\
When finished try to see if it works by typing:\\
-$>$ tclsh84\\
if ok quit by typing ctrl-c\\

\section{Compilation and installation of tk8.4.0}
Go in to the directory tk8.4.0/win\\
-$>$ cd tk8.4.0\\
-$>$ cd win\\
Then type :\\
-$>$ ./configure\\
The two steps remaining are make and make install\\
type\\
-$>$ make\\
The compilation starts, if you have errors like :\\
\\
windres -o tk.res.o --include ``C:/cygwin/home/ly/tk8.4.0/generic'' \\
--include ``C Option-I is deprecated for setting the input format,\\
 please use -J instead''\\
windres : can't open icon file 'tk.ico' : no such file or directory\\
\\
This file 'tk.ico' is in fact in the directory win/rc\\
You have to copy it in the directory 'generic':\\
Be in tk8.4.0\\
type :\\
-$>$ cp win/rc/tk.ico generic/\\
\\
If you compile again you will see that there is the same \\
errors with the files:\\
``buttons.bmp'' ``cursor00.cur'' ``cursor02.cur'' ...etc... \\
``wish.exe.manifest'' and ``wish.ico''\\
\\
Do the same for these files.\\
For the cursor*.cur files do one the command:\\
-$>$ cp win/rc/cursor*.cur generic/\\
\\
when finished install by typing:\\
-$>$ make install\\
\\
When finished try to see if it works by typing:\\
-$>$ wish84\\
if ok quit by typing ctrl-c\\

\section{Use}
Now everything is ok to use FFGUI and FreeFem++ under Windows by Cygwin.\\
WARNING: if you work on the Cygwin terminal you will not be able \\
to see the graphical results of FreeFem++.\\
\\
You have to run an X terminal and run FFGUI under this X terminal:\\
\\
To run an X terminal under cygwin type on your Cygwin terminal:\\
-$>$ startX\\
\\
An X terminal runs:\\
Under this terminal:\\
type\\
-$>$ cd FFGUI\\
-$>$ ./FreeFem++.tcl\\
\end{document}