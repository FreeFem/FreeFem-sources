\documentclass[twoside]{book}
\newif\ifpdf
\ifx\pdfoutput\undefined
\pdffalse % we are not running PDFLaTeX
\else
\pdfoutput=1 % we are running PDFLaTeX
\pdftrue
\fi
%\usepackage{times}
%\usepackage{amsmath}
\usepackage{calc}
\usepackage[latin1]{inputenc}
\usepackage{FFF}
\usepackage{amsfonts}
\usepackage{amsmath}
\usepackage{hyperref}
\usepackage{FFF}
\usepackage{makeidx}
\usepackage{color}
\usepackage{multicol}
\usepackage{graphicx}
%\usepackage{dessin}
\topmargin -1.54cm
\oddsidemargin 0cm   %marge a 2 cm
\evensidemargin 0cm  %marge a 2 cm
\newcommand{\indente}{\hbox to \parindent {\hss}}
\parindent 0cm
\headsep 0.5cm
\topskip .5cm
\footskip 1cm
\headheight 1.0cm
\textwidth  16.5cm
%  \parindent 0cm
\textheight 24cm
\def\freefempp{\texttt{freefem++ }}
\def\textRed{\color{red}}
\def\textBlack{\color{black}}
\def\Blue#1{\textcolor{blue}{#1}}
\def\Black#1{\textcolor{black}{#1}}
\def\Red#1{\textcolor{red}{#1}}
\def\Magenta#1{\textcolor{magenta}{#1}}
\def\hin{\hbox{ in }}
\def\hon{\hbox{ on }}
\def\Cpp{\texttt{C++~}}
\def\R{\mathrm{I\!R}}
\def\example{\textbf{Example:}}
\def\eq#1{\Blue{\[#1\]}}
\def\R{\mathbb{R}}
\def\Z{\mathbb{Z}}
\def\itemtt[#1]{ \item[\texttt{#1}]}
\def\plot[#1]#2#3{\begin{figure}[hbt]
\begin{center}
    \includegraphics*[#1]{#2}
\end{center}
\caption{\label{#2} #3}
\end{figure}
}
\def\Ostream{\texttt{ostream}}
\def\Istream{\texttt{istream}}
\def\Bool{\texttt{bool}}
\def\Real{\texttt{real}}
\def\Int{\texttt{int}}
\def\vecttwo#1#2{\left|\begin{smallmatrix} #1 \\ #2 \end{smallmatrix}\right.}
\def\vdeux(#1,#2){\left|\begin{smallmatrix} #1 \\ #2 \end{smallmatrix}\right.}
\def\HLINE#1{\hbox to \hsize {#1}}
\def\twoplot[#1]#2#3#4#5{
\begin{figure}[hbt]
\begin{multicols}{2}
\begin{center}
    \includegraphics*[#1]{#2}
    \caption{\label{#2} #4}
\end{center}
\begin{center}
    \includegraphics*[#1]{#3}
    \caption{\label{#3} #5}
\end{center}
\end{multicols}
\end{figure}
}% end twoplot macro
\newtheorem{remark}{\textbf{Remark}}
\newtheorem{bug}{\textbf{Bug:}}
\newtheorem{proposition}{\textbf{Proposition}}
\newtheorem{algorithm}{\textbf{Algorithm}}
\newenvironment{ttlist}
   {\begin{list}{}{\renewcommand{\makelabel}[1]{\texttt{##1}\hfil}%
        \setlength{\labelwidth}{3cm}
        \setlength{\leftmargin}{\labelwidth+\labelsep}
    }}%
   {\end{list}}


\begin{document}
\graphicspath{{./}{plots/}}
\ifpdf
\DeclareGraphicsExtensions{.pdf, .jpg, .tif}
\else
\DeclareGraphicsExtensions{.eps,.ps, .jpg}
\fi

\let\subsubsection\subsection
\let\subsection\section
\let\section\chapter


The file is formatted such that:
{\tt \obeylines
   2 nbsol nbv 2 
  $\left(\left(\mathtt{U}_{ij}, \quad \forall i \in \{1,...,\mathtt{nbsol}\}\right), \quad \forall j \in \{1,...,\mathtt{nbv}\}\right)$
 } 

where 
\begin{itemize}
\item {\tt  nbsol} is a integer equal to  the number of solutions.
\item  {\tt nbv} is  a integer equal to the number of vertex .
\item  {\tt U$_{ij}$} is a real equal the value of the $i$ solution at vertex $j$
on the associated mesh background if read file, generated if write file.
\end{itemize}
 
\subsection {BB File Type for Store Solutions}
The file is formatted such that:
{\tt \obeylines
  $ \mathtt{ \quad 2 \quad n \quad typesol^1 \quad ... \quad typesol^n \quad  nbv \quad 2}  $
  $\left(\left(\left( \mathtt{U}_{ij}^k, \quad \forall i \in \{1,...,\mathtt{typesol}^k\}\right), %
\quad \forall k \in \{1,...\mathtt{n}\}\right) %
 \quad \forall j \in \{1,...,\mathtt{nbv}\}\right)$
 } 

where 
\begin{itemize}
\item {\tt  n} is a integer equal to  the number of solutions 
\item $\mathtt{  typesol^k}$, type of the solution  number $ k$, is
  \begin{itemize}
   \item $\mathtt{typesol^k = 1}$ the solution {\tt k} is scalare  (1  value per vertex)
   \item $\mathtt{typesol^k = 2}$ the solution {\tt k} is vectorial  (2 values per unknown)
   \item $\mathtt{typesol^k = 3}$ the solution {\tt k} is a  $2\times 2$ symmetric matrix  (3 values per vertex)
   \item $\mathtt{typesol^k = 4}$ the solution  {\tt k} is a  $2\times 2$ matrix  (4 values per vertex)
   \end{itemize}

\item  {\tt nbv} is  a integer equal to the number of vertices
\item  {\tt U$_{ij}^k$} is a real equal the value of the component  $i$ of the solution  $k$ at vertex $j$
on the associated mesh background if read file, generated if write file.
\end{itemize}


\end{document}