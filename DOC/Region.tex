\documentclass[twoside]{book}
\newif\ifpdf
\ifx\pdfoutput\undefined
\pdffalse % we are not running PDFLaTeX
\else
\pdfoutput=1 % we are running PDFLaTeX
\pdftrue
\fi
%\usepackage{times}
%\usepackage{amsmath}
\usepackage{calc}
\usepackage[latin1]{inputenc}
\usepackage{FFF}
\usepackage{amsfonts}
\usepackage{amsmath}
\usepackage{hyperref}
\usepackage{FFF}
\usepackage{makeidx}
\usepackage{color}
\usepackage{multicol}
\usepackage{graphicx}
%\usepackage{dessin}
\topmargin -1.54cm
\oddsidemargin 0cm   %marge a 2 cm
\evensidemargin 0cm  %marge a 2 cm
\newcommand{\indente}{\hbox to \parindent {\hss}}
\parindent 0cm
\headsep 0.5cm
\topskip .5cm
\footskip 1cm
\headheight 1.0cm
\textwidth  16.5cm
%  \parindent 0cm
\textheight 24cm
\def\freefempp{\texttt{freefem++ }}
\def\textRed{\color{red}}
\def\textBlack{\color{black}}
\def\Blue#1{\textcolor{blue}{#1}}
\def\Black#1{\textcolor{black}{#1}}
\def\Red#1{\textcolor{red}{#1}}
\def\Magenta#1{\textcolor{magenta}{#1}}
\def\hin{\hbox{ in }}
\def\hon{\hbox{ on }}
\def\Cpp{\texttt{C++~}}
\def\R{\mathrm{I\!R}}
\def\example{\textbf{Example:}}
\def\eq#1{\Blue{\[#1\]}}
\def\R{\mathbb{R}}
\def\Z{\mathbb{Z}}
\def\itemtt[#1]{ \item[\texttt{#1}]}
\def\plot[#1]#2#3{\begin{figure}[hbt]
\begin{center}
    \includegraphics*[#1]{#2}
\end{center}
\caption{\label{#2} #3}
\end{figure}
}
\def\Ostream{\texttt{ostream}}
\def\Istream{\texttt{istream}}
\def\Bool{\texttt{bool}}
\def\Real{\texttt{real}}
\def\Int{\texttt{int}}
\def\vecttwo#1#2{\left|\begin{smallmatrix} #1 \\ #2 \end{smallmatrix}\right.}
\def\vdeux(#1,#2){\left|\begin{smallmatrix} #1 \\ #2 \end{smallmatrix}\right.}
\def\HLINE#1{\hbox to \hsize {#1}}
\def\twoplot[#1]#2#3#4#5{
\begin{figure}[hbt]
\begin{multicols}{2}
\begin{center}
    \includegraphics*[#1]{#2}
    \caption{\label{#2} #4}
\end{center}
\begin{center}
    \includegraphics*[#1]{#3}
    \caption{\label{#3} #5}
\end{center}
\end{multicols}
\end{figure}
}% end twoplot macro
\newtheorem{remark}{\textbf{Remark}}
\newtheorem{bug}{\textbf{Bug:}}
\newtheorem{proposition}{\textbf{Proposition}}
\newtheorem{algorithm}{\textbf{Algorithm}}
\newenvironment{ttlist}
   {\begin{list}{}{\renewcommand{\makelabel}[1]{\texttt{##1}\hfil}%
        \setlength{\labelwidth}{3cm}
        \setlength{\leftmargin}{\labelwidth+\labelsep}
    }}%
   {\end{list}}


\begin{document}
\graphicspath{{./}{plots/}}
\ifpdf
\DeclareGraphicsExtensions{.pdf, .jpg, .tif}
\else
\DeclareGraphicsExtensions{.eps,.ps, .jpg}
\fi

\let\subsubsection\subsection
\let\subsection\section
\let\section\chapter


This example explains the definition and manipulation of \emph{region}, i.e.
\index{subdomains} subdomains of the whole domain.

Consider this L-shaped domain with 3 diagonals as internal boundaries, defining
4 subdomains:

\bFF

//   example using region keywork
// construct a mesh with 4 regions (sub-domains)
border a(t=0,1){x=t;y=0;};
border b(t=0,0.5){x=1;y=t;};
border c(t=0,0.5){x=1-t;y=0.5;};
border d(t=0.5,1){x=0.5;y=t;};
border e(t=0.5,1){x=1-t;y=1;};
border f(t=0,1){x=0;y=1-t;};
//  internal boundary 
border i1(t=0,0.5){x=t;y=1-t;};
border i2(t=0,0.5){x=t;y=t;};
border i3(t=0,0.5){x=1-t;y=t;};
   
mesh th = buildmesh (a(6) + b(4) + c(4) +d(4) + e(4) + 
    f(6)+i1(6)+i2(6)+i3(6));
fespace Ph(th,P0);  // constant discontinuous functions / element
fespace Vh(th,P1);  // $P_1$ ontinuous functions / element

Ph reg=region; //  defined the $P_0$ function  associed to region number 
plot(reg,fill=1,wait=1,value=1);
\eFF

\twoplot[height=8cm]{region}{region_nu}{the function \texttt{reg}}{the function \texttt{nu} }

\index{region} \texttt{region}  is a keyword of freefem++ which is in fact a variable depending of 
the current position (is not a function today, use \texttt{Ph reg=region;} to  set  a function).  This variable value returned is the number of the
subdomain of the current position.  This number is defined by "buildmesh" which scans while building the mesh all
its connected component.  So to get the number of a region containing a particular point
one does:
\bFF

int nupper=reg(0.4,0.9); // get the region number of point (0.4,0.9)
int nlower=reg(0.9,0.1);  // get the region number of point (0.4,0.1)
cout << " nlower " <<  nlower << ", nupper = " << nupper<< endl;
//  defined the characteristics fonctions of upper and lower region
Ph nu=1+5*(region==nlower) + 10*(region==nupper);
plot(nu,fill=1,wait=1);
\eFF

This is particularly useful to define \index{discontinuous functions}discontinuous functions such as might occur
when one part of the domain is copper and the other one is iron, for example.
\\
We this in mind we proceed to solve a Laplace equation with discontinuous coefficients
($\nu$ is 1, 6 and 11 below).
\bFF

Ph nu=1+5*(region==nlower) + 10*(region==nupper);
plot(nu,fill=1,wait=1);
problem lap(u,v) =   int2d(th)( dx(u)*dx(v)*dy(u)*dy(v)) + int2d(-1*v) + on(a,b,c,d,e,f,u=0);
plot(u);
\eFF

\plot[height=8cm]{region_u}{the isovalue of the solution $u$}
\newpage




\end{document}