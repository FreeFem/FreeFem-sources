\documentclass[twoside]{book}
\newif\ifpdf
\ifx\pdfoutput\undefined
\pdffalse % we are not running PDFLaTeX
\else
\pdfoutput=1 % we are running PDFLaTeX
\pdftrue
\fi
%\usepackage{times}
%\usepackage{amsmath}
\usepackage{calc}
\usepackage[latin1]{inputenc}
\usepackage{FFF}
\usepackage{amsfonts}
\usepackage{amsmath}
\usepackage{hyperref}
\usepackage{FFF}
\usepackage{makeidx}
\usepackage{color}
\usepackage{multicol}
\usepackage{graphicx}
%\usepackage{dessin}
\topmargin -1.54cm
\oddsidemargin 0cm   %marge a 2 cm
\evensidemargin 0cm  %marge a 2 cm
\newcommand{\indente}{\hbox to \parindent {\hss}}
\parindent 0cm
\headsep 0.5cm
\topskip .5cm
\footskip 1cm
\headheight 1.0cm
\textwidth  16.5cm
%  \parindent 0cm
\textheight 24cm
\def\freefempp{\texttt{freefem++ }}
\def\textRed{\color{red}}
\def\textBlack{\color{black}}
\def\Blue#1{\textcolor{blue}{#1}}
\def\Black#1{\textcolor{black}{#1}}
\def\Red#1{\textcolor{red}{#1}}
\def\Magenta#1{\textcolor{magenta}{#1}}
\def\hin{\hbox{ in }}
\def\hon{\hbox{ on }}
\def\Cpp{\texttt{C++~}}
\def\R{\mathrm{I\!R}}
\def\example{\textbf{Example:}}
\def\eq#1{\Blue{\[#1\]}}
\def\R{\mathbb{R}}
\def\Z{\mathbb{Z}}
\def\itemtt[#1]{ \item[\texttt{#1}]}
\def\plot[#1]#2#3{\begin{figure}[hbt]
\begin{center}
    \includegraphics*[#1]{#2}
\end{center}
\caption{\label{#2} #3}
\end{figure}
}
\def\Ostream{\texttt{ostream}}
\def\Istream{\texttt{istream}}
\def\Bool{\texttt{bool}}
\def\Real{\texttt{real}}
\def\Int{\texttt{int}}
\def\vecttwo#1#2{\left|\begin{smallmatrix} #1 \\ #2 \end{smallmatrix}\right.}
\def\vdeux(#1,#2){\left|\begin{smallmatrix} #1 \\ #2 \end{smallmatrix}\right.}
\def\HLINE#1{\hbox to \hsize {#1}}
\def\twoplot[#1]#2#3#4#5{
\begin{figure}[hbt]
\begin{multicols}{2}
\begin{center}
    \includegraphics*[#1]{#2}
    \caption{\label{#2} #4}
\end{center}
\begin{center}
    \includegraphics*[#1]{#3}
    \caption{\label{#3} #5}
\end{center}
\end{multicols}
\end{figure}
}% end twoplot macro
\newtheorem{remark}{\textbf{Remark}}
\newtheorem{bug}{\textbf{Bug:}}
\newtheorem{proposition}{\textbf{Proposition}}
\newtheorem{algorithm}{\textbf{Algorithm}}
\newenvironment{ttlist}
   {\begin{list}{}{\renewcommand{\makelabel}[1]{\texttt{##1}\hfil}%
        \setlength{\labelwidth}{3cm}
        \setlength{\leftmargin}{\labelwidth+\labelsep}
    }}%
   {\end{list}}


\begin{document}
\graphicspath{{./}{plots/}}
\ifpdf
\DeclareGraphicsExtensions{.pdf, .jpg, .tif}
\else
\DeclareGraphicsExtensions{.eps,.ps, .jpg}
\fi

\let\subsubsection\subsection
\let\subsection\section
\let\section\chapter



Consider the problem
\begin{equation}
 -\Delta p = f \hbox{ in } \Omega=\{(x,y)\in R^2 : 0 \le x \le 1 , 0 \le y \le 1\}
\end{equation}
\begin{equation}
p = g \hbox{ on } \Gamma
\end{equation}

This problem is equivalent to the next problem : 
\begin{equation}
 u = -\nabla p 
\end{equation}
\begin{equation}
 \nabla.u = f
\end{equation}
\begin{equation}
\hbox{ in } \Omega=\{(x,y)\in R^2 : 0 \le x \le 1 , 0 \le y \le 1\}
\end{equation}
\begin{equation}
p = g \hbox{ on } \Gamma
\end{equation}

The problem is solved by the finite element method, namely:
\\
Find $u\in V$ the space of Raviart Thomas functions on a triangulation of $\Omega$ 
\begin{equation}
   \int_\Omega p*q*1e-15 + u\cdot v -p\nabla.v -q\nabla.u +f*q + \int_{\Gamma} gv\cdot n = 0
\end{equation}
The first thing to do is to prepare the mesh (i.e. the triangulation) ;
that is done by first defining the border\index{border} (the sqare) with labels one two three and four \index{label} and then call the mesh generator \index{buildmesh}(buildmesh) with the right orientation of the \index{border}border (by definition $\Omega$ is on the left side of the oriented $\Gamma$).

\bFF

mesh Th = square(10,10);
\eFF


The second thing is to define the functions spaces.

\index{fespace}
\bFF

fespace Vh(Th,RT0);  // define the Vh space
Vh [u1,u2],[v1,v2];  //  introduice the function and test function
\eFF

func f=1;
func g=1/2;
Next, \texttt{freefem++} will define
the PDE discretized by FEM in variational form with the following
instruction,  and solve the problem

\bFF

problem laplaceMixte(u1,u2,p,v1,v2,q,
solver=CG,eps=1.0e-10,tgv=1e30,
dimKrylov=150) =  
// definition of the problem
int2d(Th)( p*q*1e-15 + u1*v1 + u2*v2 
-p*(dx(v1)+dy(v2)) 
-q*(dx(u1)+dy(u2)))     //  bilinear form
+ int2d(Th)( f*q )                    //  linear form
+ int1d(Th,1)( g*(v1*N.x + v2*N.y) );  // boundary condition

laplaceMixte;  // solve the problem
plot([u1,u2],coef=0.1,wait=1,
     ps=''lapRTuv.eps'',value=true);
\eFF
 
Next we can check that the result is correct.
Here we display the result 
\begin{figure}[htbp]
	\begin{center}
		\includegraphics[scale=0.6]{LaplaceRT1.ps}
	\end{center}
	\caption{The isovalue of solution p}
\end{figure}

\begin{figure}[htbp]
	\begin{center}
		\includegraphics[scale=0.6]{LaplaceRT2.ps}
	\end{center}
	\caption{The solution $[u1,u2]$}
\end{figure}

\end{document}