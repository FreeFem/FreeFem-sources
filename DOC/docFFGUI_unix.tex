\documentclass[a4paper]{report}
\usepackage{graphicx}

\begin{document}
\chapter{The Graphic User Interface (GUI)}

\section{Installation of Tcl and Tk}

You have to install tcl8.4.6 and tk8.4.6 for the GUI to work.\\
First go to http:\\
//www.tcl.tk./software/tcltk/downloadnow84.tml \\
and download :\\
tcl8.4.6-src.tar.gz and tk8.4.6-src.tar.gz\\
Then do :\\
tar zxvf tcl8.4.6-src.tar.gz\\
tar zxvf tk8.4.6-src.tar.gz\\
It creates two directories : tcl8.4.6 and tk8.4.6\\
Now do : \\
cd tcl8.4.6\\
cd unix (if your OS is Linux or MacOSX)\\
./configure\\
make\\
make install\\
\\
then \\
cd tk8.4.6\\
cd unix\\
./configure\\
make\\
make install\\

\section{Description}

The Graphic User Interface is in the directory called FFGUI.
You can run it by typing ./FreeFem++.tcl\\
\
The Graphic User Interface shows a text window with buttons on the left and right side, a horizontal menu bar above and an entry below where you can see the path of the script when it is opened or where you can type the path of a script to run it.\\
\\
This is the description of the different functions of the GUI :\\
\\
*New : you can access it by the button ``New'' on the right side or by selecting it in the menu File on the horizontal bar.\
It deletes the current script and enables you to type a new script.\\
\\
*Open : you can access it by the button ``Open'' on the right side or by selecting it in the menu File or by typing simultaneously ctrl+o on the keyboard.\
It enables you to select a script which has been saved. This script is then opened in the text window. Then, you can modify it, save the changes, save under another name or run it.\\
\\
*Save : you can access it by the button ``Save'' on the right side or by selecting it in the menu File or by typing simultaneously ctrl+s on the keyboard.\
It enables you to save the changes of a script.\\
\\
*Save as : you can access it by the button ``Save As'' on the right side or by selecting it in the menu File.\\
It enables you to locate where you want to save your script and to choose your the name of your script.\\
\\
*Run : you can access it by the button ``Run'' on the right side or by selecting it in the menu File or by typing simultaneously ctrl+r on the keyboard.\\
It enables you to run the current script.\\
Warning: when you run by typing ctrl+r, the cursor must be in the text window.\\
\\
*Print : you can access it by the button ``Print'' on the right side or by selecting it in the menu File or by typing simultaneously ctrl+p on the keyboard.\\
It enables you to print your script. You have to choose the name of the printer.\\
\\
*Help (under construction) : you can access it by the button ``Help'' on the left side. When an example is opened, it shows you a documentation about this example.\\
Warning : The bottom of each page is not accessible, you have to print to see the entire document.\\
\\
*Example : you can access it by the button ``Ex'' on the left side.\
It runs a little example.\\
\\
*Read Mesh : you can access it by the button ``R.M'' on the left side.\
It enables you to read a mesh which has been saved. It adds the command in your script so that you can use it in your script.\\
\\
*Polygonal Border : you can access it by the button ``P.B'' on the left side.\\
It enables you to build a polygonal border.\\
When you click on this button, a new window is opened : you have to click on the button ``Border'' then it asks you how many borders you want. When you enter a number and click on ``OK'' the exact number of couple of entries enable you to enter the coordinates of the vertices. And then you have to enter the number of points on each border. The border number 1 is the segment between the vertex number 1 and the vertex number 2 ...etc...\\
You must turn in the opposite sens of needles of a watch.\\
When you have finished, you have to click on the button ``Build''. It builds the domain with polygonal border and shows the result.\\
Then you can save the result by clicking on the button ``S.Mesh''. Choose the extension .msh for the name.\\
\\
*Navier Stokes : you can access it by the button ``N.S'' on the left side.\\
A new window is opened. You have to build your polygonal domain by clicking on the button ``Border'' it works like the Polygonal Border function.\\
You can save by clicking on the button ``S.Mesh''. Choose the extension .msh for the name.\\
Then you have to choose the Limits Condition by clicking on the buton ``L.C''.\\
First choose between ``Free'' or ``Imposed', then click on the button ``Validate'' and enter the expression of Imposed condition. You have to enter the tangential and the normal component of the velocity. Then click on ``Validate'' again.\\
The expression of the limit condition can be a number but a mathematical expression as well.\\
\\
*emc2 : you can access it by the button ``emc2''. It runs emc2 the mesh building software \\
(http://www-rocq1.inria.fr/gamma/cdrom/\\
www/emc2/fra.htm)\\
\\
*Set up : you can access it by the button ``set up''. It is the first thing you have to do when you use FFGUI for the first time.\\
When you click on this button, a new window is opened, with two entries where you have to enter the path of the binary of FreeFem++ (where you compiled or installed) and the path of the scripts (programs written with FreeFem++) for instance the examples provided in FreeFem++.\\
\\
*Undo : you can access it by selecting in the menu Edit or by typing simultaneously ctrl+z. It is an unlimited undo function.\\
\\
*Redo : you can access it by selecting in the menu Edit or by typing simultaneously ctrl+e. It is an unlimited redo function.\\
\\
*Cut : you can access it by selecting in the menu Edit or by typing simultaneously ctrl+x on the keyboard.\\
\\
*Copy : you can access it by selecting in the menu Edit or by typing simultaneously ctrl+c on the keyboard.\\
\\
*Paste : you can access it by selectiog in the menu Edit or by typing simultaneously ctrl+y on the keyboard.\\
\\
*Delete : you can access it by selecting in the Edit menu. It is a Delete function.\\
\\
*Select all : you can access it by selecting in the Edit menu or by typing simultaneously ctrl+l. This function select all the text you have written, so you can delete, cut copy paste etc...\\
\\
*Background : you can access it by selecting in the menu Color. You can then choose the color of the background.\\
\\
*Foreground : you can access it by selecting in the menu Color. You can then choose the color of the foreground.\\
\\
*Syntax Color : you can access it by selecting in the menu Color. It enables the syntax coloring of your FreeFem++ script.\\
\\
*Family : you can access it by selecting in the menu Font. You can then choose the style of your characters.\\
\\
*Size : you can access it by selecting in the menu Font. You can then choose the size of your characters.\\
\\
*Find : you can access it by selecting in the menu Search. It enables you to find a word in the whole text. (This function doesn't work yet.)\\
\\
*Find next : you can access it by slecting in the menu Search. It enables you to find a word from the position of the cursor.(This function doesn't work yet.)\\
\\
*Replace : you can access it by selecting in the menu Search. It enables you to replace a word by another in the whole text.\\


\end{document}