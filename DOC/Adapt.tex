\documentclass[twoside]{book}
\newif\ifpdf
\ifx\pdfoutput\undefined
\pdffalse % we are not running PDFLaTeX
\else
\pdfoutput=1 % we are running PDFLaTeX
\pdftrue
\fi
%\usepackage{times}
%\usepackage{amsmath}
\usepackage{calc}
\usepackage[latin1]{inputenc}
\usepackage{FFF}
\usepackage{amsfonts}
\usepackage{amsmath}
\usepackage{hyperref}
\usepackage{FFF}
\usepackage{makeidx}
\usepackage{color}
\usepackage{multicol}
\usepackage{graphicx}
%\usepackage{dessin}
\topmargin -1.54cm
\oddsidemargin 0cm   %marge a 2 cm
\evensidemargin 0cm  %marge a 2 cm
\newcommand{\indente}{\hbox to \parindent {\hss}}
\parindent 0cm
\headsep 0.5cm
\topskip .5cm
\footskip 1cm
\headheight 1.0cm
\textwidth  16.5cm
%  \parindent 0cm
\textheight 24cm
\def\freefempp{\texttt{freefem++ }}
\def\textRed{\color{red}}
\def\textBlack{\color{black}}
\def\Blue#1{\textcolor{blue}{#1}}
\def\Black#1{\textcolor{black}{#1}}
\def\Red#1{\textcolor{red}{#1}}
\def\Magenta#1{\textcolor{magenta}{#1}}
\def\hin{\hbox{ in }}
\def\hon{\hbox{ on }}
\def\Cpp{\texttt{C++~}}
\def\R{\mathrm{I\!R}}
\def\example{\textbf{Example:}}
\def\eq#1{\Blue{\[#1\]}}
\def\R{\mathbb{R}}
\def\Z{\mathbb{Z}}
\def\itemtt[#1]{ \item[\texttt{#1}]}
\def\plot[#1]#2#3{\begin{figure}[hbt]
\begin{center}
    \includegraphics*[#1]{#2}
\end{center}
\caption{\label{#2} #3}
\end{figure}
}
\def\Ostream{\texttt{ostream}}
\def\Istream{\texttt{istream}}
\def\Bool{\texttt{bool}}
\def\Real{\texttt{real}}
\def\Int{\texttt{int}}
\def\vecttwo#1#2{\left|\begin{smallmatrix} #1 \\ #2 \end{smallmatrix}\right.}
\def\vdeux(#1,#2){\left|\begin{smallmatrix} #1 \\ #2 \end{smallmatrix}\right.}
\def\HLINE#1{\hbox to \hsize {#1}}
\def\twoplot[#1]#2#3#4#5{
\begin{figure}[hbt]
\begin{multicols}{2}
\begin{center}
    \includegraphics*[#1]{#2}
    \caption{\label{#2} #4}
\end{center}
\begin{center}
    \includegraphics*[#1]{#3}
    \caption{\label{#3} #5}
\end{center}
\end{multicols}
\end{figure}
}% end twoplot macro
\newtheorem{remark}{\textbf{Remark}}
\newtheorem{bug}{\textbf{Bug:}}
\newtheorem{proposition}{\textbf{Proposition}}
\newtheorem{algorithm}{\textbf{Algorithm}}
\newenvironment{ttlist}
   {\begin{list}{}{\renewcommand{\makelabel}[1]{\texttt{##1}\hfil}%
        \setlength{\labelwidth}{3cm}
        \setlength{\leftmargin}{\labelwidth+\labelsep}
    }}%
   {\end{list}}


\begin{document}
\graphicspath{{./}{plots/}}
\ifpdf
\DeclareGraphicsExtensions{.pdf, .jpg, .tif}
\else
\DeclareGraphicsExtensions{.eps,.ps, .jpg}
\fi

\let\subsubsection\subsection
\let\subsection\section
\let\section\chapter

Here we use more systematically the mesh adaptation to track the
\index{singularity}singularity at an obtuse angle of the domain.
\\
The domain is L-shaped and defined by a set of connecting segments
$a,b,c,d,e,f$ labeled $1,2,3,4,5,6$ \index{label}.
\bFF

border a(t=0,1.0){x=t;   y=0;  label=1;};
border b(t=0,0.5){x=1;   y=t;  label=2;};
border c(t=0,0.5){x=1-t; y=0.5;label=3;};
border d(t=0.5,1){x=0.5; y=t;  label=4;};
border e(t=0.5,1){x=1-t; y=1;  label=5;};
border f(t=0.0,1){x=0;   y=1-t;label=6;};
mesh Th = buildmesh (a(6) + b(4) + c(4) +d(4) + e(4) + f(6));
fespace Vh(Th,P1);
plot(Th,ps="th.eps");
\eFF

Here \texttt{plot} has an extra parameter \texttt{ps="th.eps"}.  Its effect
is to create a postscript file named "th.eps" containing the triangulation \texttt{th}
displayed during the execution of the program.
\index{postscript}
\\
Then we write the triangulation data on disk with
\index{savemesh}\texttt{savemesh} and
\texttt{th} for argument and a file name, here \texttt{th.msh}
\bFF

savemesh(th,"th.msh");  // saves mesh th in freefem format
\eFF

There are several formats available to store the mesh.
\\\\
Now we are going to solve the Laplace equation with \index{Dirichlet}
Dirichlet boundary conditions. The problem is coercive and symmetric,
so the linear system can be solved with the conjugate gradient
method \index{solver=!CG} (parameter \texttt{solver=CG}
with the stopping criteria on the residual, here
\texttt{eps=1.0e-6}).
\\ Next we solve  he same problem on an adapted (and finer) mesh 4 times:

\bFF

fespace Vh(Th,P1);  // set FE space
Vh u,v;             // set unknown and test function
real error=0.1;        // level of error
problem Probem1(u,v,solver=CG,eps=1.0e-6) =
    int2d(Th)(  dx(u)*dx(v) + dy(u)*dy(v))
  + int2d(Th) ( -v*1 )
  + on(1,2,3,4,5,6,u=0)  ;
int i;               //   declare loop index
for (i=0;i< 4;i++)
{
  Probem1;
  Th=adaptmesh(Th,u,err=error);
  error = error/2;
} ;
\eFF

after each solve a new mesh adapted to $u$ is computed.  To speed up the adaptation
we change by hand a default parameter of \texttt{adaptmesh:err}\index{concatenation}, which
specifies the required precision, so as to make the new mesh finer.

\medskip

In practice the program is more complex for two reasons

\begin{itemize}
\item We must use a \index{dynamic file names}dynamic name for files
if we want to keep track of all iterations.
This is done with the concatenation operator $+$. for instance\index{concatenation}
\bFF

for(i = 0; i< 4;i++)
    savemesh("th"+i+".msh",th);
\eFF

saves mesh th four times in files \texttt{th1.msh},\texttt{th2.msh},\texttt{th3.msh},
\texttt{th3.msh}.
\item There are many default parameters  which can be redefined either throughout the rest
of the program or locally within \texttt{adaptmesh}.
The list   with their default value is in section \ref{adaptmesh}.
\end{itemize}

\end{document}