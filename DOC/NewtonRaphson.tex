\documentclass[twoside]{book}
\newif\ifpdf
\ifx\pdfoutput\undefined
\pdffalse % we are not running PDFLaTeX
\else
\pdfoutput=1 % we are running PDFLaTeX
\pdftrue
\fi
%\usepackage{times}
%\usepackage{amsmath}
\usepackage{calc}
\usepackage[latin1]{inputenc}
\usepackage{FFF}
\usepackage{amsfonts}
\usepackage{amsmath}
\usepackage{hyperref}
\usepackage{FFF}
\usepackage{makeidx}
\usepackage{color}
\usepackage{multicol}
\usepackage{graphicx}
%\usepackage{dessin}
\topmargin -1.54cm
\oddsidemargin 0cm   %marge a 2 cm
\evensidemargin 0cm  %marge a 2 cm
\newcommand{\indente}{\hbox to \parindent {\hss}}
\parindent 0cm
\headsep 0.5cm
\topskip .5cm
\footskip 1cm
\headheight 1.0cm
\textwidth  16.5cm
%  \parindent 0cm
\textheight 24cm
\def\freefempp{\texttt{freefem++ }}
\def\textRed{\color{red}}
\def\textBlack{\color{black}}
\def\Blue#1{\textcolor{blue}{#1}}
\def\Black#1{\textcolor{black}{#1}}
\def\Red#1{\textcolor{red}{#1}}
\def\Magenta#1{\textcolor{magenta}{#1}}
\def\hin{\hbox{ in }}
\def\hon{\hbox{ on }}
\def\Cpp{\texttt{C++~}}
\def\R{\mathrm{I\!R}}
\def\example{\textbf{Example:}}
\def\eq#1{\Blue{\[#1\]}}
\def\R{\mathbb{R}}
\def\Z{\mathbb{Z}}
\def\itemtt[#1]{ \item[\texttt{#1}]}
\def\plot[#1]#2#3{\begin{figure}[hbt]
\begin{center}
    \includegraphics*[#1]{#2}
\end{center}
\caption{\label{#2} #3}
\end{figure}
}
\def\Ostream{\texttt{ostream}}
\def\Istream{\texttt{istream}}
\def\Bool{\texttt{bool}}
\def\Real{\texttt{real}}
\def\Int{\texttt{int}}
\def\vecttwo#1#2{\left|\begin{smallmatrix} #1 \\ #2 \end{smallmatrix}\right.}
\def\vdeux(#1,#2){\left|\begin{smallmatrix} #1 \\ #2 \end{smallmatrix}\right.}
\def\HLINE#1{\hbox to \hsize {#1}}
\def\twoplot[#1]#2#3#4#5{
\begin{figure}[hbt]
\begin{multicols}{2}
\begin{center}
    \includegraphics*[#1]{#2}
    \caption{\label{#2} #4}
\end{center}
\begin{center}
    \includegraphics*[#1]{#3}
    \caption{\label{#3} #5}
\end{center}
\end{multicols}
\end{figure}
}% end twoplot macro
\newtheorem{remark}{\textbf{Remark}}
\newtheorem{bug}{\textbf{Bug:}}
\newtheorem{proposition}{\textbf{Proposition}}
\newtheorem{algorithm}{\textbf{Algorithm}}
\newenvironment{ttlist}
   {\begin{list}{}{\renewcommand{\makelabel}[1]{\texttt{##1}\hfil}%
        \setlength{\labelwidth}{3cm}
        \setlength{\leftmargin}{\labelwidth+\labelsep}
    }}%
   {\end{list}}


\begin{document}
\graphicspath{{./}{plots/}}
\ifpdf
\DeclareGraphicsExtensions{.pdf, .jpg, .tif}
\else
\DeclareGraphicsExtensions{.eps,.ps, .jpg}
\fi

\let\subsubsection\subsection
\let\subsection\section
\let\section\chapter



Now, we solve the problem with Newton Ralphson algorithm, to solve the 
Euler problem $ \nabla J (u) = 0$
the algorithme is 
  $$ u^{n+1} = u^n - ( \nabla^2 J (u^{n})^{-1}*dJ(u^n) $$ 

\index{Newton}
First we introduice the two variational form \texttt{vdJ} and \texttt{vhJ} to
compute respectively $ \nabla J$ and $ \nabla^2 J$
\bFF
 
//   methode of  Newton Ralphson to solve dJ(u)=0;
//    $$ u^{n+1} = u^n - (\frac{\partial dJ}{\partial u_i})^{-1}*dJ(u^n) $$ 
//   ---------------------------------------------
  Ph dalpha ; //to store = $f''( |\nabla u|^2) $  optimisation


  // the variational form of evaluate  dJ = $ \nabla J$
  // --------------------------------------
  //  dJ =  f'()*( dx(u)*dx(vh) + dy(u)*dy(vh) 
  varf vdJ(uh,vh) =  int2d(Th)( alpha*( dx(u)*dx(vh) + dy(u)*dy(vh) ) - b*vh)
  + on(1,2,3,4, uh=0);


  // the variational form of evaluate  ddJ   $= \nabla^2 J$ 
  // hJ(uh,vh) =  f'()*( dx(uh)*dx(vh) + dy(uh)*dy(vh)
  //            + f''()( dx(u)*dx(uh) + dy(u)*dy(uh) ) * (dx(u)*dx(vh) + dy(u)*dy(vh)) 
  varf vhJ(uh,vh) = int2d(Th)( alpha*( dx(uh)*dx(vh) + dy(uh)*dy(vh) )
   +  dalpha*( dx(u)*dx(vh) + dy(u)*dy(vh)  )*( dx(u)*dx(uh) + dy(u)*dy(uh) ) )
   + on(1,2,3,4, uh=0);
   
 // the Newton algorithm
  Vh v,w; 
  u=0;
  for (int i=0;i<100;i++)
   {
    alpha = df( dx(u)*dx(u) + dy(u)*dy(u) ) ; // optimization
    dalpha = ddf( dx(u)*dx(u) + dy(u)*dy(u) ) ; // optimization
    v[]= vdJ(0,Vh);  // $ v = \nabla J(u) $
    real res= v[]'*v[]; // the dot product 
    cout << i <<  " residu^2 = " <<  res  << endl;
    if( res< 1e-12) break;
    matrix H= vhJ(Vh,Vh,factorize=1,solver=LU); //\index{matrix!factorize=}
    w[]=H^-1*v[];
    u[] -= w[];
   }
   plot (u,wait=1,cmm="solution with Newton Ralphson");
\eFF

\end{document}