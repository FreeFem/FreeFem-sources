

Consider the problem
\begin{equation}
 -\Delta v = 1 \hbox{ in } \Omega=\{(x,y)\in R^2 : x^2+y^2\le
1\}
\\
v=0\hbox{ on } \Gamma=\partial\Omega.
\end{equation}

The problem is solved by the finite element method, namely:
\\
Find $u\in V$ the space of continuous \index{femp1}piecewise linear functions
on a triangulation of $\Omega$ which are zero on the boundary
$\partial\Omega$ such that
\begin{equation}
    \int_\Omega \nabla u\cdot\nabla w = \int_\Omega w~~~\forall
    w\in V
\end{equation}
The first thing to do is to prepare the mesh (i.e. the triangulation) ;
that is done by first defining the border\index{border} (the
unit circle) with label one \index{label} and then call the mesh generator \index{buildmesh}(buildmesh) with
the right orientation of the \index{border}border (by definition $\Omega$ is
on the left side of the oriented $\Gamma$, the ).

\bFF

border a(t=0,2*pi){ x = cos(t); y = sin(t);@label=1;};
mesh disk = @buildmesh(a(50));
plot(disk);  // to see the mesh
\eFF

The second thing is to define the continuous piecewise linear functions spaces.

\index{fespace}
\bFF

fespace femp1(disk,P1);  // define the femp1 space
femp1 u,v;  //  introduice the function and test function
\eFF

Next, \texttt{freefem++} will define
the PDE discretized by FEM in variational form with the following
instruction,  and solve the problem

\bFF

  problem laplace(u,v) =  // u is the unknown and v is the test function
    int2d(disk)( dx(u)*dx(v) + dy(u)*dy(v) )     //  bilinear form
  + int2d(disk)( -1*v )                          //  linear form
  + on(1,u=0) ;                                // boundary condition

laplace;  // solve the problem
\eFF

Next we can check that the result is correct.
Here we display the result first and then display the error field
and compute the $L^2$ error and the $H^1$ error

\bFF

plot (u,value=true,wait=true); // to see the value of isoline  and wait
femp1 error=u-(1-x^2-y^2)/4;      // you only plot FE function, so do interpolation
plot(error,value=true,wait=true); // plot the error

cout << "error L2=" << sqrt(int2d(disk)( (u-(1-x^2-y^2)/4) ^2) )<< endl;
cout << "error H10=" << sqrt(   int2d(disk)((dx(u)+x/2)^2)
                              + int2d(disk)((dy(u)+y/2)^2))<< endl;
\eFF

For better results we can use mesh adaptation. This module constructs a mesh which fits
best a function of $V$, so $u$ is the main argument of \texttt{adaptmesh}.
Note that \index{adaptmesh}adaptmesh "improves" a mesh, so it requires also the name of a mesh for argument.
Therefore mesh adaptation is done in \texttt{freefem++} by
\bFF

 disk = adaptmesh(disk,u,err=0.01);
 plot(disk,wait=1);
\eFF

where \texttt{disk} is now a new mesh adapted to $u$.

To check that this mesh is better, we solve the problem again and compute the errors.
Notice the improvement!

\bFF

laplace;
plot (u,value=true,wait=true);
err =u-(1-x^2-y^2)/4;
plot(err,value=true,wait=true);
cout << "error L2=" << sqrt(int2d(disk)( (u-(1-x^2-y^2)/4) ^2) )<< endl;
cout << "error H10=" << sqrt(  int2d(disk)((dx(u)+x/2)^2)
                             + int2d(disk)((dy(u)+y/2)^2))<< endl;
\eFF

Output seen on the console:
{\small
\bFF

  Nb of common points 1
  --  mesh:  Nb of Triangles =    434, Nb of Vertices 243
   Nb of edges on Mortars  = 0
   Nb of edges on Boundary = 50, neb = 50
    Nb Mortars 0
    Number of Edges                 = 676
    Number of Boundary Edges        = 50
    Number of Mortars  Edges        = 0
    Nb Of Mortars with Paper Def    = 0 Nb Of Mortars = 0
    Euler Number nt- NbOfEdges + nv = 1= Nb of Connected Componant - Nb Of Hole
 min xy -1 -0.998027 max xy1 0.998027
 Nb Of Nodes = 243
 Nb of DF = 243
 -- Solve :           min 5.343e-32  max 0.249999
 -- borne de la function  (DF)-0.000890628 0.000858928
 min xy -1 -0.998027 max xy1 0.998027
 min xy -1 -0.998027 max xy1 0.998027
error L2=0.00211901
error H10=0.0383498
  --  mesh:  Nb of Triangles =   1535, Nb of Vertices 813
   Nb of edges on Mortars  = 0
   Nb of edges on Boundary = 89, neb = 89
    Nb Mortars 0
    Number of Edges                 = 2347
    Number of Boundary Edges        = 89
    Number of Mortars  Edges        = 0
    Nb Of Mortars with Paper Def    = 0 Nb Of Mortars = 0
    Euler Number nt- NbOfEdges + nv = 1= Nb of Connected Componant - Nb Of Hole
 min xy -0.999441 -0.999752 max xy1 0.999828
 Nb Of Nodes = 813
 Nb of DF = 813
 -- Solve :           min 3.18031e-32  max 0.249946
 min xy -0.999441 -0.999752 max xy1 0.999828
 -- function's bound   -0.000303323 0.000402198
 min xy -0.999441 -0.999752 max xy1 0.999828
error L2=0.000585005
error H10=0.0189227
\eFF
}
