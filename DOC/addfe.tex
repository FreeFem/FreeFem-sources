\section{Add new finite element}

\subsection{Some notation}
\def\Fb#1{\boldsymbol{\omega}^{K}_{#1}}
\def\fbi{\mathbf{\omega}^{K}_{ij}}


For a function $\boldsymbol{f}$ taking value in $\R^{N},\,
N=1,2,\cdots$, we define the finite element approximation $\Pi_h
\boldsymbol{f}$ of $\boldsymbol{f}$. Let us denote the number of the
degrees of freedom of the finite element by $NbDoF$. Then the $i$-th
base  $\boldsymbol{\omega}^{K}_{i}$ ($i=0,\cdots,NbDoF-1$) of the
finite element space has the $j$-th component
$\mathbf{\omega}^{K}_{ij}$ for $j=0,\cdots,N-1$.

The operator  $\Pi_{h}$ is called the interpolator of the finite element.
We have the identity $\boldsymbol{\omega}^{K}_{i} =  \Pi_{h} \boldsymbol{\omega}^{K}_{i} $.

Formally, the interpolator $\Pi_{h}$ is constructed by the following formula:
\begin{equation}
\label{eq-interpo}
\Pi_{h} \boldsymbol{f} = \sum_{k=0}^{\mathtt{kPi}-1} \alpha_k \boldsymbol{f}_{j_{k}}(P_{p_{k}}) \boldsymbol{\omega}^{K}_{i_{k}}
\end{equation}
where $P_{p}$ is a set of $npPi$ points,

%\begin{remark}
In the formula (\ref{eq-interpo}), the list $ p_{k},\, j_{k},\, i_{k}$ depend just on  the type of finite element (not on the element), but the coefficient  $\alpha_{k}$ can be depending on the element.
%\end{remark}

\medskip
%\begin{example}
 Example 1: classical scalar  Lagrange finite element, first we have $\mathtt{kPi}=\mathtt{npPi}=\mathtt{NbOfNode}$ and
\begin{itemize}
\item $P_{p}$ is the point of the nodal points
\item  the $\alpha_k=1$, because we take the value  of the function at the point $P_{k}$
\item $p_{k}=k$ ,  $j_{k}=k$ because we have one node per  function.
\item $j_{k}=0$ because $N=1$
\end{itemize}
%\end{example}

%\begin{example}
 Example 2: The Raviart-Thomas finite element:
\begin{equation}
         RT0_{h} = \{ \mathbf{v} \in H(div) / \forall K \in
         \mathcal{T}_{h} \quad  \mathbf{v}_{|K}(x,y) =
         \vecttwo{\alpha_{K}}{\beta_{K}} + \gamma_{K}\vecttwo{x}{y}  \}
         \label{eq:RT0-fe}
\end{equation}
 The degree of freedom are the flux   throw an edge $e$ of the mesh, where the flux of
 the function $\mathbf{f} : \R^2 \longrightarrow \R^2 $ is $\int_{e} \mathbf{f}.n_{e}$,
 $n_{e}$ is the unit normal of edge $e$ (this implies a orientation of all the edges of the mesh,
 for example we can use the global numbering of the edge vertices and we just go to small to large number).


  To compute this flux, we use an quadrature formula with one point, the middle point of the edge. Consider a triangle $T$ with three vertices $(\mathbf{a},\mathbf{b},\mathbf{c})$.
Let denote the  vertices numbers by $i_{a},i_{b},i_{c}$, and define the three edge vectors $\mathbf{e}^{0},\mathbf{e}^{1},\mathbf{e}^{2}$
by $ sgn(i_{b}-i_{c})(\mathbf{b}-\mathbf{c})$, $sgn(i_{c}-i_{a})(\mathbf{c}-\mathbf{a})$, $sgn(i_{a}-i_{b})(\mathbf{a}-\mathbf{b})$,

 The three basis functions are:
\begin{equation}
 \boldsymbol{\omega}^{K}_{0}= \frac{sgn(i_{b}-i_{c})}{2|T|}(x-a),\quad  \boldsymbol{\omega}^{K}_{1}= \frac{sgn(i_{c}-i_{a})}{2|T|}(x-b),\quad  \boldsymbol{\omega}^{K}_{2}= \frac{sgn(i_{a}-i_{b})}{2|T|}(x-c),
\end{equation}
where $|T|$ is the area of the triangle $T$.

So we have  $N=2$, $\mathtt{kPi}=6; \mathtt{npPi}=3;$ and:
\begin{itemize}
\item $
P_{p} = \left\{\frac{\mathbf{b}+\mathbf{c}}{2},
\frac{\mathbf{a}+\mathbf{c}}{2},
\frac{\mathbf{b}+\mathbf{a}}{2} \right\}$

\item
 $\alpha_{0}= - \mathbf{e}^{0}_{2}, \alpha_{1}= \mathbf{e}^{0}_{1}$,
 $\alpha_{2}= - \mathbf{e}^{1}_{2}, \alpha_{3}= \mathbf{e}^{1}_{1}$,
 $\alpha_{4}= - \mathbf{e}^{2}_{2}, \alpha_{5}= \mathbf{e}^{2}_{1}$ (effectively, the vector
 $ ( -\mathbf{e}^{m}_{2}, \mathbf{e}^{m}_{1}) $ is orthogonal to the edge $\mathbf{e}^{m}= (e^m_{1},e^m_{2})$ with
 a length equal to the side of the edge or equal to  $\int_{e^m} 1$).
\item $i_{k}=\{0,0,1,1,2,2\}$,
\item $p_{k}=\{0,0,1,1,2,2\}$ ,  $j_{k}=\{0,1,0,1,0,1,0,1\}$.
\end{itemize}
%\end{example}


\subsection{Which class of add}

Add file \texttt{FE\_ADD.cpp} in directory \texttt{src/femlib} for example
first to initialize :
\bFF
#include "error.hpp"
#include "rgraph.hpp"
using namespace std;
#include "RNM.hpp"
#include "fem.hpp"
#include "FESpace.hpp"

namespace  Fem2D {
\eFF

Second, you are just a class which derive for \texttt{ public  TypeOfFE} like:
\bFF
@class TypeOfFE_RTortho : public  TypeOfFE { public:
  static int Data[]; // some numbers \hfilll
  TypeOfFE_RTortho():
    TypeOfFE( 0+3+0,   // nb degree of freedom on element \hfilll
       2,      // dimension $N$  of  vectorial FE (1 if scalar FE)\hfilll
       Data,   // the array data\hfilll
       1,      // nb of subdivision for plotting\hfilll
       1,      // nb of sub finite element (generaly 1)\hfilll
       6,      // number $kPi$ of coef to build the interpolator  (\ref{eq-interpo})\hfilll
       3,      // number $npPi$ of integration point to build interpolator\hfilll
       0       // an array to store the coef $\alpha_k$ to build interpolator \hfilll
               // here this array is no constant so we have \hfilll
               // to rebuilt for each element.\hfilll
       )
  {
    const R2 Pt[] = { R2(0.5,0.5), R2(0.0,0.5), R2(0.5,0.0) };
    // the set of Point in $\hat{K}$
    for (int p=0,kk=0;p<3;p++) {
      P_Pi_h[p]=Pt[p];
      for (int j=0;j<2;j++)
        pij_alpha[kk++]= IPJ(p,p,j); }} // definition of $i_{k},p_{k},j_{k}$ in (\ref{eq-interpo})

  void FB(const bool * watdd, const Mesh & Th,const Triangle & K,
          const R2 &PHat, RNMK_ & val) const;

  void Pi_h_alpha(const baseFElement & K,KN_<double> & v) const ;

} ;
\eFF
where  the array data is form with the concatenation of  five array of size \texttt{NbDoF} and one
array of size \texttt{N}.

This array is:
\bFF
@int TypeOfFE_RTortho::Data[]={

              // for each df 0,1,3 :  \hfilll
        3,4,5,// the support of the node of the df   \hfilll
        0,0,0,// the number of the df on  the node   \hfilll
        0,1,2,// the node of the df  \hfilll
        0,0,0,// the df come from which FE (generally 0) \hfilll
        0,1,2,// which are de df on sub FE \hfilll
        0,0 };// for each component $j=0,N-1$ it give the sub FE associated
\eFF
where the support is a number $0,1,2$ for vertex support, $3,4,5$ for edge support,
and finaly $6$ for element support.


The function to defined the function $\boldsymbol{\omega}^{K}_{i}$, this function return
the value of all the basics function or this derivatives in array
\texttt{val}, computed at point \texttt{PHat} on the reference triangle corresponding
to point \texttt{R2 P=K(Phat);} on the current triangle \texttt{K}.

The index $i,j,k$ of the array $val(i,j,k)$   corresponding to:
\begin{description}
\item[$i$]  is basic function number on finite element  $i \in [0,NoF[ $
\item[$j$]  is the value of component   $ j \in [0,N[ $
\item[$k$]  is the type of computed value $f(P),dx(f)(P), dy(f)(P), ...$
$i \in [0,\mathtt{last\_operatortype}[ $. Remark for optimization, this value is computed only if  $whatd[k]$ is true, and the numbering is defined with
\bFF
@enum operatortype { op_id=0,
   op_dx=1,op_dy=2,
   op_dxx=3,op_dyy=4,
   op_dyx=5,op_dxy=5,
   op_dz=6,
   op_dzz=7,
   op_dzx=8,op_dxz=8,
   op_dzy=9,op_dyz=9
   };
const int last_operatortype=10;
\eFF
\end{description}

The shape function :
\bFF
 void TypeOfFE_RTortho::FB(const bool *whatd,const Mesh & Th,const Triangle & K,
                           const R2 & PHat,RNMK_ & val) const
{ //
  R2 P(K(PHat));
  R2 A(K[0]), B(K[1]),C(K[2]);
  R l0=1-P.x-P.y,l1=P.x,l2=P.y;
  assert(val.N() >=3);
  assert(val.M()==2 );
  val=0;
  R a=1./(2*K.area);
  R a0=   K.EdgeOrientation(0) * a ;
  R a1=   K.EdgeOrientation(1) * a  ;
  R a2=   K.EdgeOrientation(2) * a ;

  //  ------------
  @if (whatd[op_id])  // value of the function
   {
     @assert(val.K()>op_id);
     RN_ f0(val('.',0,0)); // value first component
     RN_ f1(val('.',1,0)); // value second component
     f1[0] =  (P.x-A.x)*a0;
     f0[0] = -(P.y-A.y)*a0;

     f1[1] =  (P.x-B.x)*a1;
     f0[1] = -(P.y-B.y)*a1;

     f1[2] =  (P.x-C.x)*a2;
     f0[2] = -(P.y-C.y)*a2;
    }
  // ----------------
    @if (whatd[op_dx]) // value of the dx of function
    {
     assert(val.K()>op_dx);
     val(0,1,op_dx) =  a0;
     val(1,1,op_dx) =  a1;
     val(2,1,op_dx) =  a2;
     }
    @if (whatd[op_dy])
    {
     assert(val.K()>op_dy);
     val(0,0,op_dy) =  -a0;
     val(1,0,op_dy) =  -a1;
     val(2,0,op_dy) =  -a2;
    }

  @for (@int i= op_dy; i< last_operatortype ; i++)
   @if (whatd[op_dx])
     @assert(op_dy);

}
\eFF

The function to defined the coefficient $\alpha_{k}$:
\bFF
void TypeOfFE_RT::Pi_h_alpha(const baseFElement & K,KN_<double> & v) const
{
  const Triangle & T(K.T);

   for (int i=0,k=0;i<3;i++)
     {
        R2 E(T.Edge(i));
        R signe = T.EdgeOrientation(i) ;
        v[k++]= signe*E.y;
        v[k++]=-signe*E.x;
     }
}
\eFF

Now , we just need to add a new key work in \texttt{FreeFem++}, so
at the end of the file, we add:

\bFF
//  let the 2 globals variables
static TypeOfFE_RTortho The_TypeOfFE_RTortho; //
//                         -----  the name in freefem ----
static  ListOfTFE typefemRTOrtho("RT0Ortho", & The_TypeOfFE_RTortho); //

// link with FreeFem++  do not work with static library .a \hfilll
//  FH so add a extern name to call in \texttt{init\_static\_FE} \hfilll
// (see end of FESpace.cpp) \hfilll
void init_FE_ADD() { };
// --- end --  \hfilll
} // FEM2d namespace
\eFF

To inforce in loading of this new finite element,
we have to add the two new lines close to the end of files \texttt{src/femlib/FESpace.cpp}
like:
\bFF
// correct Problem of static library link with new make file
void init_static_FE()
{ //  list of other FE file.o
   extern void init_FE_P2h() ;
  init_FE_P2h() ;
   extern void init_FE_ADD() ;  // new line 1
   init_FE_ADD();  // new line 2
}

\eFF


\subsection{How to add}

First, create a file \texttt{FE\_ADD.cpp} contening all this code, like in  file \texttt{src/femlib/Element\_P2h.cpp},
after modifier the \texttt{Makefile.am}  by  adding the name of your file
to the variable \texttt{EXTRA\_DIST} like:

\begin{verbatim}
# Makefile using Automake + Autoconf
# ----------------------------------
# $Id$

# This is not compiled as a separate library because its
# interconnections with other libraries have not been solved.

EXTRA_DIST=BamgFreeFem.cpp BamgFreeFem.hpp CGNL.hpp CheckPtr.cpp        \
ConjuguedGradrientNL.cpp DOperator.hpp Drawing.cpp Element_P2h.cpp      \
Element_P3.cpp Element_RT.cpp fem3.hpp fem.cpp fem.hpp FESpace.cpp      \
FESpace.hpp FESpace-v0.cpp FQuadTree.cpp FQuadTree.hpp gibbs.cpp        \
glutdraw.cpp gmres.hpp MatriceCreuse.hpp MatriceCreuse_tpl.hpp          \
MeshPoint.hpp mortar.cpp mshptg.cpp QuadratureFormular.cpp              \
QuadratureFormular.hpp RefCounter.hpp RNM.hpp RNM_opc.hpp RNM_op.hpp    \
RNM_tpl.hpp   FE_ADD.cpp

\end{verbatim}

and recompile



For codewarrior compilation add the file in the project an remove the flag
in panal  PPC linker FreeFEm++ Setting Dead-strip Static Initializition Code Flag.
